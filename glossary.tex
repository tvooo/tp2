
% Acronyms

% Products
\newacronym{ide}{IDE}{Integrated Development Environment}
\newacronym{wysiwyg}{WYSIWYG}{What You See Is What You Get}
\newacronym{parc}{PARC}{Palo Alto Research Centre}
\newacronym{oss}{OSS}{Open Source Software}

% Components
\newacronym{gui}{GUI}{Graphical User Interface}
\newacronym{ui}{UI}{User Interface}
\newacronym{api}{API}{Application Programming Interface}

% Paradigms and stuff
\newacronym{ucd}{UCD}{User-Centered Design}
\newacronym{jit}{JIT}{Just-In-Time}
\newacronym{hci}{HCI}{Human-Computer-Interaction}
\newacronym{qa}{QA}{Quality Assurance}
\newacronym{ast}{AST}{Abstract Syntax Tree}
\newacronym{apm}{APM}{Atom Package Manager}

% Web stuff
\newacronym{dom}{DOM}{Document Object Model}
\newacronym{xml}{XML}{Extensible Markup Language}
\newacronym{json}{JSON}{JavaScript Object Notation}
\newacronym{html}{HTML}{Hypertext Markup Language}
\newacronym{css}{CSS}{Cascading Stylesheets}
\newacronym{js}{JS}{JavaScript}
\newacronym{ajax}{AJAX}{Asynchronous JavaScript and XML}

% Technologies
% Server
\newacronym{http}{HTTP}{Hypertext Transfer Protocol}
\newacronym{tcp}{TCP}{Transmission Control Protocol}
\newacronym{asp}{ASP}{Active Server Pages}
\newacronym{php}{PHP}{PHP Hypertext Preprocessor}
\newacronym{jsp}{JSP}{JavaServer Pages}
\newacronym{cgi}{CGI}{Common Gateway Interface}
\newacronym{jee}{Java EE}{Java Platform, Enterprise Edition}
\newacronym{ejb}{EJB}{Enterprise JavaBeans}
\newacronym{url}{URL}{Uniform Resource Locator}
\newacronym{jre}{JRE}{Java Runtime Environment}
\newacronym{sql}{SQL}{Structured Query Language}
\newacronym{crud}{CRUD}{Create, Read, Update, Delete}
\newacronym{jvm}{JVM}{Java Virtual Machine}
\newacronym{amd}{AMD}{Asynchronous Module Definition}
\newacronym{ria}{RIA}{Rich Internet Application}

% Others
\newacronym{ietf}{IETF}{Internet Engineering Taskforce}
\newacronym{w3c}{W3C}{World Wide Web Consortium}
\newacronym{ibm}{IBM}{International Business Machines Corporation}
\newacronym{gps}{GPS}{Global Positioning System}
\newacronym{saas}{SaaS}{Software--as--a--Service}
\newacronym{paas}{PaaS}{Platform--as--a--Service}
\newacronym{iaas}{IaaS}{Infrastructure--as--a--Service}
\newacronym{io}{I/O}{Input/Output}
\newacronym{rest}{REST}{Representational State Transfer}
\newacronym{sso}{SSO}{Single Sign-On}
\newacronym{oop}{OOP}{Object-Oriented Programming}

% Glossary

\newglossaryentry{syntaxhighlighting}
{
	name=syntax highlighting,
	description={Highlighting of programming language elements (such as reserved words, identifiers, or certain data types) using text formatting and colour, for the purpose of better readability and understanding.}
}

\newglossaryentry{identifier}
{
	name=identifier,
	description={A lexical token that names a programming language entity, such as a variable or class.}
}

\newglossaryentry{variable}
{
	name=variable,
	description={A storage location and an associated symbolic name (identifier) which contains a value (data).}
}

\newglossaryentry{function}
{
	name=function,
	description={A sequence of program instructions that perform a specific task, packaged as a unit.}
}

\newglossaryentry{node}
{
	name=Node.js,
	description={A server-side platform to write JavaScript applications.}
}

\newglossaryentry{webkit}
{
	name=WebKit,
	description={The layout engine used in browsers such as Chrome and Safari.}
}

\newglossaryentry{javascript}
{
	name=JavaScript,
	description={A dynamic computer programming language, most commonly used as part of web browsers.}
}

\newglossaryentry{ecmascript}
{
	name=ECMAScript,
	description={The language standard (ISO/IEC 16262) implemented by JavaScript.}
}

\newglossaryentry{this}
{
	name=this,
	description={The JavaScript keyword \texttt{this} refers to the current execution context of a program during run-time.}
}

\newglossaryentry{breadcrumbs}
{
	name=breadcrumbs,
	description={A user interface pattern that acts as navigation aid, allowing users to keep track of their location and navigate upwards in the hierarchy.}
}

\newglossaryentry{codesmell}
{
	name=code smell,
	description={A symptom in the source code of a program that possibly indicates a deeper problem, although not being technically incorrect (in opposition to a bug).}
}

\newglossaryentry{coffeescript}
{
	name=CoffeeScript,
	description={A programming language that transcompiles to JavaScript.}
}

\newglossaryentry{staticanalysis}
{
	name=static analysis,
	description={The analysis of computer software that is performed without actually executing programs.}
}

\newglossaryentry{autocompletion}
{
	name=autocompletion,
	description={Predicting a word or phrase that the user wants to type in without the user actually typing it in completely.}
}

